\documentclass[12pt]{article}

\begin{document}


\section*{Chapter 2 - Computer Architecture}

\subsection*{Know the Concepts}
\begin{enumerate}
    \item Describe the fetch execute cycle.
        \textbf{Answer:}
    \item What is a register? How would computation be more difficult without a register?
        \textbf{Answer:}
    \item How do you represent numbers larger than 255?
        \textbf{Answer:}
    \item How big are the registers on the machines we will be using?
        \textbf{Answer:}
    \item How does a computer know how to interpret a given byte or set of bytes in memory.
        \textbf{Answer:}
    \item What are the addressing modes and what are they used for?
        \textbf{Answer:}
    \item What does the instruction pointer do?
        \textbf{Answer:}
\end{enumerate}

\subsection*{Use the Concepts}
\begin{enumerate}
    \item What data would you use in an employee record? How would you lay it out in memory?
        \textbf{Answer:}
    \item If I had a pointer to the beginning of the employee record above, and wanted to access a particular ouece of data inside of it, what addressing mode would I use?
        \textbf{Answer:}
    \item In base pointer addressing mode, if you have a register holding the value 3122, and an offset of 20, what address would you be trying to access?
        \textbf{Answer:}
    \item In indexed addressing mode, if the base address is 6512, the index register has a 5, and the multiplier is 4, what address would you be trying to access?
        \textbf{Answer:}
    \item In indexed addressing mode, if the base address is 123472, the index register has a 0, and the multiplier is 4, what address would you be trying to access?
        \textbf{Answer:}
    \item In indexed addressing mode, if the base address is 9123478, the index register has a 20, and the multiplier is 1, what address would you be trying to access?
        \textbf{Answer:}
\end{enumerate}

\subsection*{Going Further}
\begin{enumerate}
    \item What is the minimum number of addressing modes needed for computation?
        \textbf{Answer:}
    \item Why include addressing modes that aren't strictly needed?
        \textbf{Answer:}
    \item Research and then describe how pipelining (or one of the other complicating factors) affects the fetch-execute cycle.
        \textbf{Answer:}
    \item Research and then describe the tradeoffs between fixed-length instructions and variable-length instructions.
        \textbf{Answer:}
\end{enumerate}


\end{document}
